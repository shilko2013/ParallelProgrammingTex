\documentclass{article}

\usepackage[utf8]{inputenc}
\usepackage[russian]{babel}
\usepackage[left=3cm, right=2.5cm, top=2cm, bottom=2cm]{geometry}
\usepackage{graphicx}
\usepackage{pscyr}
\renewcommand{\rmdefault}{ftm}
\graphicspath{ {./images/} }

\pagestyle{empty}
\parindent=1.5cm

\begin{document}
	\begin{flushright}
		\includegraphics[width=9cm, height=1.5cm]{ITMOLogo2}
	\end{flushright}
	\Large\par\textbf{Миссия университета} – генерация передовых знаний, внедрение инновационных разработок и подготовка элитных кадров, способных действовать в условиях быстро меняющегося мира и обеспечивать опережающее развитие науки, технологий и других областей для содействия решению актуальных задач.
	\newline
	\noindent\rule{\textwidth}{1.5pt} %horizontal line
	\newline
	\newline
	\textbf{\centerline{КАФЕДРА ВЫЧИСЛИТЕЛЬНОЙ ТЕХНИКИ}}
	\newline
	\parКафедра вычислительной техники Университета ИТМО более 70 лет ведёт подготовку высококвалифицированных специалистов в области информатики и вычислительной техники. За годы своего существования на кафедре подготовлено более 5,5 тысяч высококвалифицированных специалистов, подготовлено более 240 кандидатов наук и 37 докторов наук. Преподавателями и научными сотрудниками кафедры изданы десятки монографий, учебников и учебно-методических пособий. Кафедра имеет высококвалифицированный состав преподавателей, среди которых 7 профессоров и 14 доцентов, обучающих более 500 студентов и аспирантов.
	\newline
	\parБолее 15 лет кафедра ведёт подготовку по двухуровневой схеме: бакалавриат-магистратура. За эти годы подготовлено около полутора тысяч бакалавров и магистров по направлению «Информатика и вычислительная техника». В 2011 году на кафедре началась подготовка бакалавров и магистров по новому направлению «Программная инженерия». 
	\newline
	\parВедущими преподавателями кафедры разработаны 7 магистерских программ, охватывающих наиболее актуальные направления вычислительной техники, такие как вычислительные системы и компьютерные сети, информационная безопасность и защита информации, встроенные вычислительные системы и системы на кристалле, программные комплексы и информационные интеллектуальные системы. Магистранты кафедры вычислительной техники активно участвуют в научно-исследовательских разработках кафедры, выступают с докладами на конференциях и семинарах различных уровней, публикуют свои результаты в научно-техничес-ких журналах и сборниках.
\end{document}