{ %section2_4
	\subsection{Модификация закона Амдала (по проф. Бухановскому)}
	\parВ реальных вычислительных системах ОС тратит ресурсы на создание и удаление новых потоков. Время, затраченное на эти операции не учитывается в законе Амдала. Параллельное ускорение $S(p)$ зависит от количества ядер и доли распараллеливаемых операций, но не зависит от количества последних. Выведем формулу в которой количество операций для которых необходимо создать поток будет учитываться.
	\parПусть $N$ – количество распараллеливаемых операций, $M$ – количество нераспараллеливаемых операций, $t_c$ – время выполнения одной операции, $p$ – количество вычислителей(ядер), $T_i$ – время выполнения программы при использовании $i$ параллельных потоков на $i$ вычислителях, $\alpha$ – некий масштабирующий коэффициент, инкапсулирующий в себе количество времени, требуемого на создание, удаление потока и прочие накладные операции. 
По формуле~\eqref{AmdalSFromP:equation}, $S(p)\;=\;\frac{T_1}{T_p}$.
	\parНайдем сначала $T_1$. Так как это код выполняется линейно, то время затраченное на его выполнение будет равно количеству операций помноженному на время выполнения одной операции: $T_1\;=\;t_c(N\;+\;M)$. 
	\parВремя выполнение распараллельнной программы $T_p$ включается в себя время на создание потока: $t_c\alpha(p\;-\;1)N$ (нужно создать $(p\;-\;1)$ новых потоков, так как главный поток уже создан и для каждого затратить какое-то время $\alpha$), время работы распараллеливаемоего кода на всех ядрах: $\frac {t_cN}p$ и время работы нераспараллеливаемого кода $t_cM$. Итого, разделив $T_1$ на $T_p$, получим формулу закона Амдала по проф. Бухановскому:
	\begin{equation}
		\label{AmdalBuhunovsky:equation}
		S(p,N)\;=\;\frac{T_1}{T_p}\;=\;\frac{N\;+\;M}{\alpha(p\;-\;1)N\;+\;\frac Np\;+\;M}
	\end{equation}
	Из формулы~\eqref{AmdalBuhunovsky:equation} видно, что с ростом количество ядер после определенного предела $S(p,N)$  не будет расти как в законе Амдала, так как время будет тратиться много времени на создание новых потоков. На рисунке~\ref{GraphAmdalBuhunovsky:image} наглядно видно, что $S(p,N)$ уменьшается при большом количестве потоков и становится заметно меньше $S(p)$ по Амдалу даже при небольшом значении $\alpha$.
	\begin{figure}[H]
		\includegraphics[width=1\linewidth]{GraphAmdalBuhunovsky}
		\caption{\textit{График зависимости параллельного ускорения от количества потоков}}
		\label{GraphAmdalBuhunovsky:image}
	\end{figure}
}