{ %section3_2
	\subsection{Менеджеры управления памятью для параллельных программ}
	\parПри вызове функций malloc/free в однопоточной программе не возникает проблем даже при довольно высокой интенсивности вызовов одной из них. Однако в параллельных программах эти функции могут стать узким местом, т.к. при их одновременном использовании из нескольких потоков происходит блокировка общего ресурса (менеджера управления памятью), что может привести к существенной деградации скорости работы многопоточной программы.
	\parПолучается, что несмотря на формальную потокобезопасность стандартных функций работы с памятью, они могут стать потоконеэффективными при очень интенсивной работе с памятью нескольких параллельно работающих потоков.
	\parДля решения этой проблемы существует ряд сторонних программ, называющихся ''Менеджер управления памятью (МУП)'' (Memory Allocator), как платных, так и бесплатных с открытым исходным кодом. Каждое из них обладает своими достоинствами и недостатками, которые следует учитывать при выборе. Перечислим наиболее распространённые МУП с указанием ссылок на официальные сайты:
	\begin{itemize}
		\sloppy
		\item tcmalloc: \url{http://goog-perftools.sourceforge.net/doc/tcmalloc.html}
		\item ptmalloc: \url{http://www.malloc.de/malloc/ptmalloc3-current.tar.gz}
		\item dmalloc: \url{http://dmalloc.com/}
		\item HOARD: \url{http://www.hoard.org/}
		\item nedmalloc: \url{http://www.nedprod.com/programs/portable/nedmalloc/}
		\item jemalloc: \url{http://jemalloc.net/}
		\item mimalloc: \url{https://github.com/microsoft/mimalloc}
	\end{itemize}
	\parПеречисленные МУП разработаны таким образом, что ими можно ''незаметно'' для параллельной программы подменить стандартные МУП библиотеки libc языка С. Это значит, что выбор конкретного МУП никак не влияет на исходный код программы, поэтому общая практика использования сторонних МУП такова: параллельная программа изначально создаётся с использованием МУП libc, затем проводится профилирование работающей программы, затем при обнаружении узкого места (bottleneck) в функциях malloc/free принимается решение заменить стандартный МУП одним из перечисленных.
	\parТакже стоит отметить, что некоторые технологии распараллеливания (например, Intel TBB) уже имеют в своём составе специализированный МУП, оптимизированный для выполнения в многопоточном режиме.
	\par
}