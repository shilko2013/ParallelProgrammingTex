{ %section3_3
	\subsection{Библиотека Intel IPP}
	\label{IPP:section}
	\par\textbf{Оптимизация типовых задач обработки данных.} Существует немногочисленное количество высокопроизводительных библиотек, состоящих из набора низкоуровневых API для обработки данных: изображений, сигналов, матриц.
	\parОдной из таких библиотек является Intel Integrated Performance Primitives(Intel IPP), реализующия следущие функции:
	\begin{itemize}
		\itemКодирование и декодирование видео.
		\itemКодирование и декодирование аудио.
		\itemJPEG/JPEG2000.
		\itemКомпьютерное зрение.
		\itemКриптография.
		\itemСжатие данных.
		\itemПреобразование цвета.
		\itemОбработка изображения.
		\itemТрассировка луча/визуализация.
		\itemОбработка сигналов.
		\itemКодирование речи.
		\itemРаспознование речи.
		\itemОбработка строк.
		\itemВекторная/матричная математика.
	\end{itemize}
	Для использования функций данной библиотеки необходимо в исходном коде подключить заголовочный файл IPP:
	\begin{figure}[H]
		\lstinputlisting{includeIPP.c}
	\end{figure}
	\parРассмотрим пример программы которая вычисляет модуль синуса каждого элемента массива:
	\begin{figure}[H]
		\lstinputlisting{withoutIPP.c}
	\end{figure}
	\parТеперь воспользуемся функциями IPP, тогда наша программа будет выглядеть так:
	\begin{figure}[H]
		\lstinputlisting{withIPP.c}
	\end{figure}
    \parБлагодаря использованию данных функция, программа стала компактнее и быстрее.
    \parВсе о использовании функций IPP можно узнать из официальной документации \url{https://software.intel.com/content/www/us/en/develop/documentation/ipp-dev-reference/top.html}.
    \par
}