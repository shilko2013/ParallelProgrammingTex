{ %section3_1
	\subsection{Отладка параллельных программ}
	\parСредства отладки параллельных программ встроены в большинство популярных интегрированных сред разработки (IDE), например: Visual Studio, Eclipse CDT, Intel Parallel Studio и т.п. Эти средства включают в себя удобную визуализацию временных диаграмм исполнения потоков, автоматический поиск подозрительных участков программы, в которых могут наблюдаться гонки данных и взаимоблокировки.
	\parНесмотря на эффективность существующих инструментов отладки, при работе в дебаггере (debugger) с параллельной программой возникают существенные затруднения, т.к. для своего корректного функционирования отладчик добавляет в машинный код исходной параллельной программы дополнительные инструкции, которые изменяют временную диаграмму выполнения потоков по отношению друг к другу. Это может приводить к ситуациям, когда при тестировании программы в отладчике не наблюдаются гонки данных и взаимоблокировки, которые при запуске Release-версии программы проявятся в полной мере.
	\parТакже при отладке многопоточной программы следует иметь в виду, что её поведение (как при штатной работе, так и при отладке) может существенным образом различаться при использовании одноядерного и многоядерного процессора. При запуске нескольких потоков на одноядерной машине они будут выполняться в режиме разделения времени, т.е. последовательно. Значит, в этом случае не будут наблюдаться многие проблемы с совместным доступом к памяти и обеспечением когерентности кэшей, присущие многоядерным системам. Кроме того, при отладке программы на одноядерной системе программист может использовать неявные приёмы обеспечения последовательности выполнения операций. 
	\parНапример, программист может некорректно предполагать, что при выполнении высокоприоритетного потока низкоприоритетный поток не может завладеть процессором. Это предположение корректно только в одноядерной системе, ведь при наличии нескольких ядер и малом количестве высокоприоритетных потоков вполне может наблюдаться ситуация, когда низкоприоритетный поток завладеет одним из ядер, при одновременной работе высокоприоритетного потока на соседнем ядре.
	\par
}