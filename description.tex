{ %description
	\thispagestyle{empty} % delete number of  page
	\parСоснин В.В., Балакшин П.В. Введение в параллельные вычисления. – СПб: Университет ИТМО, 2019. – 51 с.
	\newline
	\newline
	\parВ пособии излагаются основные понятия и определения теории параллельных вычислений. Рассматриваются основные принципы построения программ на языке «Си» для многоядерных и многопроцессорных вычислительных комплексов с общей памятью. Предлагается набор заданий для проведения лабораторных и практических занятий.
	\newline
	\newline
	\parУчебное пособие предназначено для студентов, обучающихся по магистерским программам направления «09.04.04 – Программная инженерия», и может быть использовано выпускниками (бакалаврами и магистрантами) при написании выпускных квалификационных работ, связанных с проектированием и исследованием многоядерных и многопроцессорных вычислительных комплексов.
	\newline
	\newline
	\parРекомендовано к печати Ученым советом факультета компьютерных технологий и управления, 8 декабря 2015 года, протокол №10.\textbf{НЕТ!!!!}
	\vspace*{\fill} %text in the bottom of page
	\begin{flushright}
		\includegraphics[width=9cm, height=1.5cm]{ITMOLogo2}
	\end{flushright}
	\parУниверситет ИТМО – ведущий вуз России в области информационных и фотонных технологий, один из немногих российских вузов, получивших в 2009 году статус национального исследовательского университета. С 2013 года Университет ИТМО – участник программы повышения конкурентоспособности российских университетов среди ведущих мировых научно-образовательных центров, известной как проект «5 в 100». Цель Университета ИТМО – становление исследовательского университета мирового уровня, предпринимательского по типу, ориентированного на интернационализацию всех направлений деятельности.
	\begin{flushright}
		\copyright\spaceУниверситет ИТМО, 2019
		\par\copyright\spaceСоснин В.В., Балакшин П.В., 2019
	\end{flushright}
}